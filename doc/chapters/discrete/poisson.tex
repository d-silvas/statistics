\section{Description}
The Poisson distribution expresses the probability of a given number of events occurring in a fixed interval of time or space if these events occur with a known constant mean rate and independently of the time since the last event. The Poisson distribution can also be used for the number of events in other specified intervals such as distance, area or volume.

For instance, an individual keeping track of the amount of mail they receive each day may notice that they receive an average number of 4 letters per day. If receiving any particular piece of mail does not affect the arrival times of future pieces of mail, i.e., if pieces of mail from a wide range of sources arrive independently of one another, then a reasonable assumption is that the number of pieces of mail received in a day obeys a Poisson distribution. Other examples that may follow a Poisson distribution include the number of phone calls received by a call center per hour, the number of decay events per second from a radioactive source, the number of visits to a website and the number of typographical errors per page in a book.

The Poisson distribution is an appropriate model if the following assumptions are true:

\begin{itemize}
	\item k is the number of times an event occurs in an interval and k can take values 0, 1, 2, ....
	\item The occurrence of one event does not affect the probability that a second event will occur. That is, events occur independently.
	\item The average rate at which events occur is constant.
	\item Two events cannot occur at exactly the same instant; instead, at each very small sub-interval exactly one event either occurs or does not occur.
\end{itemize}

The Poisson distribution can also be developed as a limiting distribution of the binomial, in which $n \rightarrow \infty$ and $p \rightarrow 0$ so that $np$ remains a constant. In other words, for large $n$ and small $p$, the binomial distribution can be approximated by the Poisson distribution with mean $\lambda = np$.
If the sample was drawn without replacement from a small finite population, the hypergeometric distribution should be used instead of the binomial.

If the number of events per unit time follows a Poisson distribution, then the amount of time between events follows the exponential distribution.

\subsection{Probability mass function}
Let $X$ denote the number of events in a unit interval of time or in a unit distance. Then, $X$ is called the Poisson random variable with mean number of events $\lambda$ in a unit interval of time. The probability mass function of a Poisson distribution with mean $\lambda$ is given by
\[
 	f(k \mid \lambda) = P(X = k \mid \lambda) = \frac{e^{-\lambda} \lambda^k}{k!}, \ k = 0, 1, 2, \ldots
\]

The Poisson probability mass function is right-skewed, because it is inhibited by the zero occurrence barrier (there can not happen less than 0 events) on the left and it is unlimited on the other side.
The degree of skewness decreases as $\lambda$ increases. As $\lambda$ becomes bigger, the graph looks more like a normal distribution.

\begin{figure}[H]
	\centering
	\begin{subfigure}[b]{0.45\textwidth}
		\includegraphics[width=\textwidth]{discrete/poisson/pmf_2.pdf}
		\caption{$\lambda = 2$}
	\end{subfigure}
	\begin{subfigure}[b]{0.45\textwidth}
		\includegraphics[width=\textwidth]{discrete/poisson/pmf_3.pdf}
		\caption{$\lambda = 3$}
	\end{subfigure}
	\begin{subfigure}[b]{0.45\textwidth}
		\includegraphics[width=\textwidth]{discrete/poisson/pmf_5.pdf}
		\caption{$\lambda = 5$}
	\end{subfigure}
	\begin{subfigure}[b]{0.45\textwidth}
		\includegraphics[width=\textwidth]{discrete/poisson/pmf_10.pdf}
		\caption{$\lambda = 10$}
	\end{subfigure}
	\begin{subfigure}[b]{0.45\textwidth}
		\includegraphics[width=\textwidth]{discrete/poisson/pmf_20.pdf}
		\caption{$\lambda = 20$}
	\end{subfigure}
	\begin{subfigure}[b]{0.45\textwidth}
		\includegraphics[width=\textwidth]{discrete/poisson/pmf_30.pdf}
		\caption{$\lambda = 30$}
	\end{subfigure}
	\caption{Poisson distribution}
\end{figure}

\begin{figure}[H]
	\includegraphics[width=\textwidth]{discrete/poisson/pmf_all.pdf}
	\caption{$f(k \mid \lambda) = P(X = k \mid \lambda) = \frac{e^{-\lambda} \lambda^k}{k!}, \ k = 0, 1, 2, \ldots$}
\end{figure}

\subsection{Cumulative distribution function}
\[
	F(k \mid \lambda) = P(X = k \leq \lambda) = \sum_{i = 0}^{k} \frac{e^{-\lambda} \lambda^i}{i!}, \ k = 0, 1, 2, \ldots
\]

\begin{figure}[H]
	\centering
	\begin{subfigure}[b]{0.45\textwidth}
		\includegraphics[width=\textwidth]{discrete/poisson/cdf_2.pdf}
		\caption{$\lambda = 2$}
	\end{subfigure}
	\begin{subfigure}[b]{0.45\textwidth}
		\includegraphics[width=\textwidth]{discrete/poisson/cdf_3.pdf}
		\caption{$\lambda = 3$}
	\end{subfigure}
	\begin{subfigure}[b]{0.45\textwidth}
		\includegraphics[width=\textwidth]{discrete/poisson/cdf_5.pdf}
		\caption{$\lambda = 5$}
	\end{subfigure}
	\begin{subfigure}[b]{0.45\textwidth}
		\includegraphics[width=\textwidth]{discrete/poisson/cdf_10.pdf}
		\caption{$\lambda = 10$}
	\end{subfigure}
	\begin{subfigure}[b]{0.45\textwidth}
		\includegraphics[width=\textwidth]{discrete/poisson/cdf_20.pdf}
		\caption{$\lambda = 20$}
	\end{subfigure}
	\begin{subfigure}[b]{0.45\textwidth}
		\includegraphics[width=\textwidth]{discrete/poisson/cdf_30.pdf}
		\caption{$\lambda = 30$}
	\end{subfigure}
	\caption{Poisson distribution}
\end{figure}

\begin{figure}[H]
	\includegraphics[width=\textwidth]{discrete/poisson/cdf_all.pdf}
	\caption{$F(k \mid \lambda) = P(X = k \leq \lambda) = \sum_{i = 0}^{k} \frac{e^{-\lambda} \lambda^i}{i!}, \ k = 0, 1, 2, \ldots$}
\end{figure}

\section{Moments}

\begin{tabularx}{\textwidth}{s X}
	\hline
	Mean & $\lambda$ \\\hline
	Variance & $\lambda$\\\hline
\end{tabularx}

\section{Examples}

\begin{example}[Approximation of the binomial to a Poisson]
	(Taken from \cite{AerinKim})

	Suppose we have a website. Each person that visits the website has some probability of clicking an ad. The binomial distribution can help us calculate the probability of successful events (clicks).
	A binomial random variable is the number of successes ($x$) in $n$ repeated trials. We assume the probability of success $p$ is constant over each trial.

	We are interested in knowing the number of people that will click an ad per week. Let's assume we have the stats for a year: a total of 59000 poeple visited the website, and out of them, 888 people clicked an ad.
	Then, $n = 59000/12 = 1134$. The number of people who clicked an ad per week is $888/52 = 17$. The probability of success is $p = 888/59000 = 0.015$.

	At this point we can calculate, for example, the probability that next week exactly 20 people click an add. We use the binomial pmf for that:
	\[
		P(X = 20) = \binom{n}{x} p^x (1 - p)^{n - x} = \binom{1134}{20} \ 0.015^20 \ (1 - 0.015)^{1134 - 20} = 0.06962
	\]

	Other values are as follows:

	\begin{tabular}{c | c}
		x & Binomial $P(X = x)$ \\\hline
		10 & 0.02250 \\\hline
		17 & 0.09701 \\\hline
		20 & 0.06962 \\\hline
		30 & 0.00121 \\\hline
		40 & $< 0.000001$ \\
	\end{tabular}

	Note that the \textbf{expected value} or \textbf{mean}, 17 (which is the average number of successes per week we calculated from the raw data), has the highest probability of happening.
\end{example}

% TODO: http://www.sascha-frank.com/Faq/tables_four.html
% TODO: https://www.overleaf.com/learn/latex/Environments