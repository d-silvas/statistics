\section{Distribution functions}

In this chapter, we will consider probability spaces of the form $(\R, \B, \P)$, where the only variable is the probability function $\P$. It is posible to characterize any probability
function $\P$ in $\R$ via a real-valued real function:

\begin{definition}
	A \textbf{distribution function} is a function $F: \R \rightarrow [0, 1]$ such that:
	\begin{enumerate}
		\item $F$ increases in $\R$: $F(x_1) \leq F(x_2)$ for all $x_1 < x_2$.
		\item $F$ is right-continuous: for every $x \in \R$, $F(x) = \lim_{y \rightarrow x, y > x} F(y)$.
		\item $\lim_{y \rightarrow -\infty} F(x) = 0$ and $\lim_{y \rightarrow \infty} F(x) = 1$.
	\end{enumerate}
\end{definition}

Condition 1 guarantees the existence of both one-sided limits:
\[
	F(x^+) = \lim_{y \rightarrow x, y > x} F(y) \text{  and  } F(x^-) = \lim_{y \rightarrow x, y < x} F(y)
\]
From condition 2, we have that $F(x^+) = F(x)$. However, at some points, we could have that $F(x^-) \neq F(x)$. In this case, $F$ presents a jump discontinuity at $x$.

% TODO examples
TODO Examples...

Any probability fnuction $\P$ in $(\R, \B)$ determines a distribution function:

\begin{prop}
	Let $\P$ be a probability function in $(\R, \B)$. Then, the function $F$ defined as follows:
	\[
		F(x) = \P((-\infty, x])
	\]
	is a distribution function.
\end{prop}