%\begin{figure}[H]
%	\centering
%	\includegraphics[width=\textwidth]{pr0201.png}
%\end{figure}
\subsection{Description}
A binomial experiment involves $n$ independent and identical trial such that each trial can result into one of the two possible outcomes: success of failure. If $p$ is the probability of observing success in each trial, then the number of successes $X$ that can be observed out of these $n$ trials is referred to as the \textbf{binomial random variable with $n$ trials and success probability $p$}.

\subsubsection{Probability mass function}
The probability of observing $k$ successes out of $n$ trials is given by the following probability mass function
\[
	P(X = k \mid n, p) = \binom{n}{k}p^{k}(1 - p)^{n - k}, \quad k = 0, 1, \ldots, n
\]

\subsubsection{Cumulative distribution function}
\[
	P(X \leq k \mid n, p) = \sum_{i = 0}^{k} \binom{n}{i}p^{i}(1 - p)^{n - i}, \quad k = 0, 1, \ldots, n
\]

\subsection{Moments}

\subsection{Plots}

\subsection{Examples}
